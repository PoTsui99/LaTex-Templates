\documentclass{CV}
\usepackage{ctex}
\usepackage{indentfirst}
\setname{崔}{博}
\setaddress{吉林大学|软件学院}
\setmobile{(+86)17361825184}
\setmail{tsuipo@outlook.com}
\setposition{Work Student} %ignored for now
\setlinkedinaccount{ }
\setgithubaccount{https://github.com/PoTsui99} 
\setthemecolor{CornflowerBlue}
\begin{document}
%Set variables

%Create header
\headerview
\vspace{1ex}
%Sections
%
% Summary - We will not use Summary as it is a waste of space.
% \addblocktext{Summary}{%
% \lipsum[1][1-12] %replace this part with actual text
% }

%
% Experience
中共预备党员,江苏南通籍;大二学年获国家励志奖学金;曾多次参与数学类竞赛;曾多次参与机器学习相关项目并顺利结题;对传统机器学习以及深度学习算法有一定研究与实践;对 NLP 以及 CV 方向研究感兴趣.
\section{教育经历} 

\noindent\datedexperience{吉林大学-软件学院-软件工程专业}{本科生涯}{2018-2022} 
\explanationdetail{
	\coloredbullet\ % 
	\textbf{GPA:} 3.53/4.0 \quad{} \quad{} \quad{}
 \textbf{专业排名:}  23/355(6.5\%)\\[0.1cm]
	\coloredbullet\ % 
	课程成绩:线代(94分);概率论(94分);程序设计基础(90分);数据结构(90);计组(91分);操作系统(90分);数据库(91分)\\[0.1cm]
	\coloredbullet\ 
	荣誉奖项:国家励志奖学金(2020.12);吉林大学二等奖学金(2020.04); 吉林大学院优秀学生(2020.04)
}

\section{竞赛经历}
\noindent\mydatedexperience{美国大学生数学建模竞赛(ICM)D题 Meritorious奖}{2021.04} 
\mydatedexperience{全国大学生数学建模竞赛(本科组)吉林赛区二等奖}{2020.11}
\mydatedexperience{吉林省数学竞赛(非数学专业)三等奖}{2020.11}
\mydatedexperience{吉林大学"互联网+"大学生创新创业大赛赛道优秀奖}{2020.11}

\section{研究经历}

\noindent\datedexperience{}{鸟鸣识别与图像匹配}{2021.03-2021.05}
    \explanationdetail{
    	\coloredbullet\ % 
    对标注好类别的鸟鸣数据建立模型并进行训练,进而对连续音频中的鸟鸣进行识别并\textbf{匹配对应图像}\\[0.1cm]
    	\coloredbullet\ % 
    担任项目组队长,主要负责数据获取,模型训练以及鸟鸣预测\\[0.1cm]
    	\coloredbullet\ % 
    提取音频梅尔对数频谱特征,利用\textbf{ResNeXt}预训练模型以及Adam优化器进行训练\\[0.1cm]
    	\coloredbullet\ % 
    采用多模型投票集成,相比单模型提高了约14\%的准确度
    }
		
\noindent\mydatedexperience{随机森林特征选择算法优化}{2020.09-2020.11} 
	\explanationdetail{
		\coloredbullet\ % 
		对传统的随机森林特征选择算法(FSFOA)优化论文进行复现并进行优化\\[0.1cm]
		\coloredbullet\ % 
		担任项目组核心成员,主要承担算法改进工作\\[0.1cm]
		\coloredbullet\ % 
		优化初始化过程(控制特征数量);优化局部播种过程(使新树fitness value不低于原树);优化全局播种范围(将范围扩大到全体年龄为0的树木)\\[0.1cm]
		\coloredbullet\ % 
		最后在wine、lonosphere、heart 等数据集上的预测准确度\textbf{提高了1.5\%到5.1\%}
	}

\noindent\mydatedexperience{瓦斯突出模式的模糊识别}{2020.10-2020.12} 
	\explanationdetail{
		\coloredbullet\ % 
		对瓦斯突出模式模糊识别论文进行复现\\[0.1cm]
		\coloredbullet\ % 
		担任项目组队长,主要承担算法编写工作\\[0.1cm]
		\coloredbullet\ % 
		首先通过对样本间\textbf{汪培庄贴进度}的计算得到相似度矩阵;再找到合理的阈值去”截”相似度矩阵\\[0.1cm]
		\coloredbullet\ % 
		将样本分为中高低等不同风险等级,继而对任一样本给出的特征数值,能够判断其大致的瓦斯突出风险
	}
%    \newcommand{\projectone}{%
%    \lipsum[1][7-8]\par %replace this part with actual text
%    }
%    %
%    \newcommand{\projecttwo}{%
%    \lipsum[1][9-10]\par %replace this part with actual text
%    }
%    %
%    \newcommand{\projectthree}{%
%    \lipsum[1][11-12]%replace this part with actual text
%    }
%    %
%    \newcommand{\listofprojects}{\projectone, \projecttwo, \projectthree}
%    %
%    \createbullets{\listofprojects}    
    
%
% Additional Experience and Awards
\section{校园经历}
%    \newcommand{\extraone}{%
%    参加向日葵志愿者社团,看望过敬老院老人及自闭症儿童
%    }
%    %
%    \newcommand{\extratwo}{%
%    参加学校马克思毛泽东思想学研会
%    }
%    %
%%    \newcommand{\extrathree}{%
%%    \lipsum[1][11-12]%replace this part with actual text
%%    }
%    %
%    \newcommand{\listofextras}{\extraone, \extratwo
%%    	, \extrathree
%    }
%    %
%    \createbullets{\listofextras}    
\noindent\mydatedexperiencetwo{参加向日葵志愿者社团,看望过敬老院老人及自闭症儿童}{2019}	
\noindent\mydatedexperiencetwo{参加吉林大学校马克思毛泽东思想学研会}{2018}	
%
% Skills
\section{技能及其它}
    %
    \newcommand{\skillone}{\createskill{编程技能}{C++(面向对象及算法实现) \cpshalf Python(机器学习) \cpshalf Django(后端开发) \cpshalf Java(软构件及中间件基础) \cpshalf R/Matlab(灵活运用开源代码)}}
    %
    \newcommand{\skilltwo}{\createskill{语言水平}{CET-4获596分 \cpshalf CET-6获537分 \cpshalf 四学期英语满绩}}
    %
    \newcommand{\skillthree}{\createskill{日常爱好}{阅读文史哲书籍 \cpshalf 跑步和羽毛球等运动}}
    %
%    \newcommand{\skillfour}{\createskill{iOS Programming}{RxSwift \cpshalf PromiseKit \cpshalf CocoaPods \cpshalf Autolayout/DSLs}}
%    %
%    \newcommand{\skillfive}{\createskill{Languages}{\textbf{\emph{Native:}} \ \  Turkish \ \ \textbf{\emph{Fluent:}} \ \ English \ \ \textbf{\emph{Beginner:}} \ \  German }}
    %
    \createskills{\skillone, \skilltwo, \skillthree
    }

%
%Footnote
\createfootnote
\end{document}