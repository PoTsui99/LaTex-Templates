\documentclass{CV}
\usepackage{ctex}
\usepackage{indentfirst}
\setname{崔}{博}
\setaddress{北京航空航天大学·计算机学院(2022级保研生)}
\setmobile{(+86)17361825184}
\setmail{tsuipo@outlook.com}
\setposition{Work Student} %ignored for now
%\setlinkedinaccount{ }
%\setgithubaccount{https://github.com/PoTsui99} // 反注释 cv.cls.\linkedinview
\setthemecolor{CornflowerBlue}
\begin{document}
\fangsong
%Set variables

%Create header
\headerview
\vspace{1ex}
%Sections
%
% Summary - We will not use Summary as it is a waste of space.
% \addblocktext{Summary}{%
% \lipsum[1][1-12] %replace this part with actual text
% }

%
% Experience
\vspace{5pt}
\vspace*{10pt}
籍贯江苏南通, 本科曾获国家励志奖学金, 目前研究方向为遥感图像目标检测. 对传统机器学习以及深度学习算法有一定了解与掌握, 对 NLP 以及 CV 方向研究感兴趣.

%\vspace*{7pt}
\vspace{15pt}

\section{\heiti 教育经历} 

\vspace*{5pt}

\noindent
\hspace*{-2pt}\datedexperience{\heiti 吉林大学-软件学院-软件工程}{\heiti 本科生涯}{2018-2022} 
\explanationdetail{
	\coloredbullet\ % 
	\textbf{GPA:} 3.59/4.0 \quad{} \quad{} \quad{}
 \textbf{专业排名(前六学期必修课):}  23/355(6.5\%)\\[0.1cm]
	\coloredbullet\ % 
	课程成绩: 线性代数(94); 概率论与数理统计(94); 程序设计基础(90); 数据结构(90); 计算机组成原理(91); 操作系统(90); 数据库(91)\\[0.1cm]
	\coloredbullet\ 
	荣誉奖项: 国家励志奖学金(2020.12); 吉林大学二等奖学金(2020.04);  吉林大学院优秀学生(2020.04)
}

\vspace*{5pt}

\noindent
\hspace*{-2pt}\datedexperience{\heiti 北京航空航天大学大学-计算机学院-计算机技术}{\heiti 硕士生涯}{2022-2025} 
\explanationdetail{
%	\coloredbullet\ % 
%	\textbf{GPA:} 3.59/4.0 \quad{} \quad{} \quad{}
%	\textbf{专业排名(前六学期必修课):}  23/355(6.5\%)\\[0.1cm]
	\coloredbullet\ % 
	课程成绩: 矩阵理论(98); 人工智能原理与应用(92); 遥感图像解译(88)\\[0.1cm]
%	\coloredbullet\ 
%	荣誉奖项: 国家励志奖学金(2020.12); 吉林大学二等奖学金(2020.04);  吉林大学院优秀学生(2020.04)
}


\vspace*{-12pt}
\section{\heiti 竞赛经历}
\vspace*{5pt}
\noindent
\hspace*{-2pt}\mydatedexperience{美国大学生数学建模竞赛(ICM)D题 Meritorious奖}{2021.04} 
\mydatedexperience{全国大学生数学建模竞赛(本科组)吉林赛区二等奖}{2020.11}
%\mydatedexperience{吉林省数学竞赛(非数学专业)三等奖}{2020.11}
%\mydatedexperience{吉林大学"互联网+"大学生创新创业大赛赛道优秀奖}{2020.11}

\vspace*{-4pt}
\section{\heiti 研究经历}
\vspace*{5pt}

\noindent
\hspace*{-2pt}\datedexperience{}{\heiti 鸟鸣识别与图像匹配}{2021.03-2021.05}
%\vspace{2pt}
    \explanationdetail{
    	\coloredbullet\ % 
    对标注好类别的鸟鸣数据建立模型并进行训练, 进而对连续音频中的鸟鸣进行识别并\textbf{匹配对应图像}\\[0.1cm]
    	\coloredbullet\ % 
    担任项目组队长, 主要负责数据获取,模型训练以及鸟鸣预测\\[0.1cm]
    	\coloredbullet\ % 
    提取音频梅尔对数频谱特征,利用\textbf{ResNeXt}预训练模型以及Adam优化器进行训练\\[0.1cm]
    	\coloredbullet\ % 
    采用多模型投票集成, 相比单模型准确度提高了约14\%
    }
		
%\vspace*{-10pt}
%\noindent
%\mydatedexperience{随机森林特征选择算法优化}{2020.09-2020.11} 
%	\explanationdetail{
%		\coloredbullet\ % 
%		对传统的随机森林特征选择算法(FSFOA)优化论文进行复现并进行优化\\[0.1cm]
%		\coloredbullet\ % 
%		担任项目组核心成员,主要承担算法改进工作\\[0.1cm]
%		\coloredbullet\ % 
%		优化初始化过程(控制特征数量);优化局部播种过程(使新树fitness value不低于原树);优化全局播种范围(将范围扩大到全体年龄为0的树木)\\[0.1cm]
%		\coloredbullet\ % 
%		最后在wine、lonosphere、heart 等数据集上的预测准确度\textbf{提高了1.5\%到5.1\%}
%	}

\vspace*{-10pt}
\noindent
\mydatedexperience{\heiti 瓦斯突出模式的模糊识别}{2020.10-2020.12} 
	\explanationdetail{
		\coloredbullet\ % 
		对瓦斯突出模式模糊识别论文进行复现\\[0.1cm]
		\coloredbullet\ % 
		担任项目组队长,主要承担算法编写工作\\[0.1cm]
		\coloredbullet\ % 
		首先通过对样本间汪培庄贴进度的计算得到相似度矩阵; 再找到合理的阈值去”截”相似度矩阵\\[0.1cm]
		\coloredbullet\ % 
		将样本分为中高低等不同风险等级,继而对任一样本给出的特征数值,能够判断其大致的瓦斯突出风险
	}

\vspace*{-10pt}
\noindent
\mydatedexperience{\heiti 基于U-Net的脑肿瘤影像分割(本科毕设)}{2022.02-2022.05} 
\explanationdetail{
	\coloredbullet\ % 
	优化U-Net及U-Net++网络结构并在BraTS数据集上进行训练与评估\\[0.1cm]
%	\coloredbullet\ % 
%	担任项目组队长,主要承担算法编写工作\\[0.1cm]
	\coloredbullet\ % 
	向U-Net中添加批量归一化层, 对特征图进行Same Padding; 使用ResBlock作为U-Net++的Backbone\\[0.1cm]
	\coloredbullet\ % 
	各分割区域的Dice相关系数、Hausdorff距离等指标有显著提高
}
%    \newcommand{\projectone}{%
%    \lipsum[1][7-8]\par %replace this part with actual text
%    }
%    %
%    \newcommand{\projecttwo}{%
%    \lipsum[1][9-10]\par %replace this part with actual text
%    }
%    %
%    \newcommand{\projectthree}{%
%    \lipsum[1][11-12]%replace this part with actual text
%    }
%    %
%    \newcommand{\listofprojects}{\projectone, \projecttwo, \projectthree}
%    %
%    \createbullets{\listofprojects}    
    
%
% Additional Experience and Awards
\vspace*{5pt}
%\section{校园经历} 
%\noindent\hspace*{-2pt}\mydatedexperiencetwo{参加向日葵志愿者社团,看望过敬老院老人及自闭症儿童}{2019}	
%\noindent\mydatedexperiencetwo{参加吉林大学校马克思毛泽东思想学研会}{2018}	
%%
%% Skills
%\vspace*{-15pt}
%\vspace*{-3pt}
\section{\heiti 技能及其它}
\vspace*{5pt}

    %
    \newcommand{\skillone}{\createskill{编程技能}{C++(面向对象及算法实现) \cpshalf Python(机器学习) \cpshalf Java(软构件及中间件基础) \cpshalf R/Matlab(简单掌握)}}
    %
    \newcommand{\skilltwo}{\createskill{语言水平}{CET-4: 596分 \cpshalf CET-6: 537分 \cpshalf 本科英语课程全部满绩}}
    %
    \newcommand{\skillthree}{\createskill{日常爱好}{读书 \cpshalf 跑步和羽毛球等运动}}
    \newcommand{\skillfour}{\createskill{性格特点}{擅长内省 \cpshalf 富有团队合作精神}}
    %
%    \newcommand{\skillfour}{\createskill{iOS Programming}{RxSwift \cpshalf PromiseKit \cpshalf CocoaPods \cpshalf Autolayout/DSLs}}
%    %
%    \newcommand{\skillfive}{\createskill{Languages}{\textbf{\emph{Native:}} \ \  Turkish \ \ \textbf{\emph{Fluent:}} \ \ English \ \ \textbf{\emph{Beginner:}} \ \  German }}
    %
    \hspace*{-30pt}\createskills{\skillone, \skilltwo, \skillthree, \skillfour
    }

%
%Footnote
\vspace*{-28pt}
\createfootnote
\end{document}