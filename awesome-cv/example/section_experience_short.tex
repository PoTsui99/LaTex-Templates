\sectionTitle{专业知识}{\faTasks}
\par{\quad{}}
%\renewcommand{\labelitemi}{$\bullet$}
\begin{experiences}
  \experience
    {编程语言}{课内外学习的高级程序语言}{
          \begin{itemize}
            \item C/C++:较为熟练掌握,掌握面向对象编程范式.手动实现过Huffman树,关键路径,SPFA等算法.
            \item Python:接触较多框架,能够灵活用于机器学习,数据挖掘及简易WEB项目开发.
            \item Java:掌握基础知识,能应用软构件及中间件技术开发项目.
            \item R/Matlab:看得懂代码,能够按需修改开源代码.
          \end{itemize}}
  \emptySeparator
  \experience
  {机器学习}{传统机器学习及深度学习}{
  	\begin{itemize}
  		\item 具备相对扎实的微积分,线性代数,概率论与数理统计知识.
  		\item 能够使用sklearn,librosa等Python库进行缺失值填补,transform等预处理.
  		\item 能够使用预训练的模型解决当前问题,掌握调参技巧,了解深度学习常用trick.
%  		\item 程序设计基础(C语言)
%  		\item 数字逻辑
%  		\item 离散数学
%  		\item 面向对象程序设计(C++语言)
%  		\item 数据结构
%  		\item 计算机组成原理(双语)
%  		\item 算法分析
%  		\item 操作系统原理
%  		\item 数据库系统原理
%  		\item 计算机网络(双语)
%  		\item Software System Analysis and Design
%  		\item Software Analysis
%  		\item 统一建模语言及工具(双语)
%  		\item 软件设计模式
%  		\item 编译原理与实现
  \end{itemize}}
  \emptySeparator
  \experience
  {其余知识}{计算机基础知识}{
  	\begin{itemize}
  		\item 具备离散数学相关知识,对集合论,群论,图论,数理逻辑,运筹学等领域的基础问题有所认知.
  		\item 掌握计算机组成原理,操作系统等硬件及软硬件边缘的计算机知识,掌握Linux基础的command.
  		\item 能够运用UML,软件系统分析设计等知识,对软件系统架构进行分析与设计,掌握软件开发测试相关的方法论.
  \end{itemize}}
  \emptySeparator
\end{experiences}
